\documentclass[t,14pt,mathserif]{beamer}


\usepackage[portuguese]{babel}
\usepackage[utf8]{inputenc}

\usepackage[T1]{fontenc}
\fontfamily{cmr}\selectfont

\usepackage{ragged2e}
\usepackage[flushleft]{threeparttable}
\usepackage{tabularx}

\usepackage{graphicx}
\graphicspath{ {./desktop/} }

\usepackage{blindtext}
\usepackage{booktabs,mathrsfs}
\usepackage{tcolorbox}
\usepackage{amsmath}



\documentclass[t,14pt,mathserif]{beamer}


\usepackage[portuguese]{babel}
\usepackage[utf8]{inputenc}

\usepackage[T1]{fontenc}
\fontfamily{cmr}\selectfont

\usepackage{ragged2e}
\usepackage[flushleft]{threeparttable}
\usepackage{tabularx}

\usepackage{graphicx}
\graphicspath{ {./desktop/} }

\usepackage{blindtext}
\usepackage{booktabs,mathrsfs}
\usepackage{tcolorbox}
\usepackage{amsmath} 



% Copyright 2012 by Aécio S. R. Santos <aecio.solando@gmail.com>.
%
% In principle, this file can be redistributed and/or modified under
% the terms of the GNU Public License, version 2.
%
% However, this file is supposed to be a template to be modified
% for your own needs. For this reason, if you use this file as a
% template and not specifically distribute it as part of a another
% package/program, I grant the extra permission to freely copy and
% modify this file as you see fit and even to delete this copyright
% notice.

% Redefines the font type
\usepackage{helvet}
\usefonttheme[onlymath]{serif}

% Some useful colors
\definecolor{lightblue}{rgb}{.0, .68, .84}
\definecolor{black}{rgb}{0, 0, 0}
\definecolor{gray}{rgb}{0.3, 0.3, 0.3}

%
% BLUE color scheme
%
\newcommand{\setcolorschemeblue}{
	\definecolor{titlecolor}{rgb}{0, 0.37, 0.59}
	\definecolor{bulletscolor}{rgb}{.0, .68, .84}
	\definecolor{alertcolor}{rgb}{.0, .68, .84}
}

%
% PURPLE color scheme
%
% Purple color pallete
% base 0.35, 0.2, 0.55 (between h2 and h3)
% h1   0.21, 0.1, 0.34 (darker)
% h2   0.28, 0.2, 0.45
% h3   0.44, 0.4, 0.70
% h4   0.51, 0.2, 0.71 (lighter)
%
\newcommand{\setcolorschemepurple}{
	\definecolor{titlecolor}{rgb}{0.35, 0.2, 0.55}
	\definecolor{bulletscolor}{rgb}{0.44, 0.4, 0.70}
	\definecolor{alertcolor}{rgb}{0.35, 0.2, 0.55}
}

%
% GREEN color scheme
%
\newcommand{\setcolorschemegreen}{
	\definecolor{titlecolor}{rgb}{0, 0.5, 0.48}
	\definecolor{bulletscolor}{rgb}{0.2,0.2,0.7}
	\definecolor{alertcolor}{rgb}{.0, .68, .84}
}

% Define default colors scheme
\setcolorschemeblue

% Define color of alert text
\setbeamercolor{alerted text}{fg=alertcolor}

% block environment
%\setbeamertemplate{blocks}[rounded=true, shadow=true]
\setbeamertemplate{blocks}[rounded][shadow=true]
\setbeamercolor*{block body}{fg=black,bg=titlecolor!20!white}
%\setbeamercolor*{block title}{fg=titlecolor!70!black,bg=titlecolor!40!white}
\setbeamercolor*{block title}{fg=titlecolor!10!white,bg=titlecolor!75!white}
%\setbeamerfont{block title}{size=\large,series=\bf}

% Define font sizes
\setbeamerfont{frametitle}{parent=structure,size=\large}
\setbeamerfont{framesubtitle}{parent=frametitle,size=\footnotesize}
\setbeamerfont{itemize/enumerate body}{size=\fontsize{16pt}{17.6pt}}
\setbeamerfont{itemize/enumerate subbody}{size=\fontsize{14pt}{15,4pt}}
\setbeamerfont{itemize/enumerate subsubbody}{size=\footnotesize}

% Define font sizes for bibliography
\setbeamerfont{bibliography entry author}{size=\small}
\setbeamerfont{bibliography entry title}{size=\small}
\setbeamerfont{bibliography entry location}{size=\small}
\setbeamerfont{bibliography entry note}{size=\small}

% Redefine the cover title fonts to be bold
\setbeamerfont{title}{size=\large, series=\bfseries}

% Redefine cover title color
\setbeamercolor{title}{fg=titlecolor}

% Redefine title color
\setbeamercolor{frametitle}{fg=titlecolor,size=20pt}

% Uncomment to redefine bullets with round format
\useinnertheme[shadow]{rounded}
\setbeamertemplate{blocks}[rounded][shadow=\beamer@themerounded@shadow]
\setbeamertemplate{items}[ball]

% Redefine table of content colors
\setbeamercolor{section in toc}{fg=titlecolor}
\setbeamercolor{subsection in toc}{fg=titlecolor!40!black}

% Redefine bibliography colors
\setbeamercolor{bibliography entry author}{fg=titlecolor!25!black}
\setbeamercolor{bibliography entry title}{fg=titlecolor}
\setbeamercolor{bibliography entry location}{fg=titlecolor!25!black}

% Redefine bullets color
\setbeamercolor*{item}{fg=bulletscolor}

% Redefine spacing of left margin of bullets
\setlength{\leftmargini}{1.3em}
\setlength{\leftmarginii}{1em}
\setlength{\leftmarginiii}{1em}

% Redefine space between of items in 'itemize' enviroment
\newlength{\wideitemsep}
\setlength{\wideitemsep}{\itemsep}
\addtolength{\wideitemsep}{0.25pt}
\let\olditem\item
\renewcommand{\item}{\setlength{\itemsep}{\wideitemsep}\olditem}

% Redefine space before a nested itemize
\makeatletter
\def\@listii{\leftmargin\leftmarginii
	\topsep    0.9ex
	\parsep    0\p@   \@plus\p@
	\itemsep   \parsep}
\makeatother


% Redefine width of text area margins
\setbeamersize{text margin left=1em,text margin right=1em}

% Define summary items depth
\setcounter{tocdepth}{2}

% Redefine styles of frames' title
\setbeamertemplate{frametitle} {
	\vspace{0.2cm}
	\ifbeamercolorempty[bg]{frametitle}{}{\nointerlineskip}%
	\begin{beamercolorbox}[]{frametitle}
		\ifbeamercolorempty[bg]{frametitle}{}{\nointerlineskip}%
		\usebeamerfont{frametitle}{%
			\strut\insertframetitle\strut\par%
		}
		{%
			\ifx\insertframesubtitle\@empty%
			\else
			\usebeamerfont{framesubtitle}\usebeamercolor[fg]{framesubtitle}\insertframesubtitle\strut\par
			\fi
			\vspace{-.9cm}%
			{
				\textcolor{gray} {\rule[5pt]{\linewidth}{.5pt}\vspace{-8pt}}
			}
		}%
		\vskip-0.5ex%
		\if@tempswa\else\vskip-.9cm\fi
	\end{beamercolorbox}%
	\vspace{0.2cm}
}

% Removes navigation bar
\beamertemplatenavigationsymbolsempty

% Redefine footline to show only slide number
\setbeamertemplate{footline}{
	\begin{beamercolorbox}[wd=1\paperwidth,ht=2.25ex,dp=1ex,right]{date in head/foot}%
		%\hfill
		\insertframenumber{}                             % Only current slide number
		%\insertframenumber{} / \inserttotalframenumber  % Current slide number and total of slides
		\hspace{2ex}
	\end{beamercolorbox}
}




\title[ ] % (optional, use only with long paper titles)
{Análise de Regressão Múltipla com Informações Qualitativas: \\ Variáveis Binárias (ou Dummy)}

\subtitle
{Wooldridge Cap. 7}


\author[shortname]
{Cleiton Guollo Taufemback \inst{1} 
\and \\ Miguel Jandrey Natal \inst{2}}

\institute[shortinst]{\inst{1} affiliation for author1 \and %
                      \inst{2} affiliation for author2}
\institute[Universidade Federal do Rio Grande do Sul] % (optional, but mostly needed)
{
	\inst{1}%
	Departamento de Estatística\\ 
	\and \inst{2} %
    Mestrando em estágio docência (PPGEst)\\
    \and
    
	Universidade Federal do Rio Grande do Sul
	
	}

\date[ ] 


\begin{document}

\begin{frame}
\titlepage
\end{frame}



\begin{frame}{A Descrição das Informações Qualitativas}
		\justifying
		\vfill
		Nos capítulos anteriores, as variáveis dependentes e independentes em nossos modelos de regressão múltipla tinham significado quantitativo. No trabalho empírico também devemos incorporar fatores qualitativos nos modelos de regressão.
		\vfill
		
         O sexo ou raça de um indivíduo, o ramo de atividade de uma empresa (fabricante, varejista, etc.) e a região onde uma cidade está localizada (sul, norte, oeste etc.) são todos considerados fatores qualitativos.
		\vfill
	

\end{frame}


\begin{frame}{A Descrição das Informações Qualitativas}
		\justifying
		\vfill
		Fatores qualitativos frequentemente aparecem na forma de informação binária: uma pessoa é do sexo feminino ou masculino; alguém possui ou não um computador pessoal, etc. Em todos esses exemplos, a informação relevante pode ser capturada pela definição de uma variável binária (também chamada de $dummy$) ou uma variável zero-um.
		\vfill
		
         Ao definirmos uma variável $dummy$, precisamos decidir a qual evento será atribuído o valor um e a qual será atribuído o valor zero.
		\vfill
	

\end{frame}

\begin{frame}{A Descrição das Informações Qualitativas}
		\justifying
		\vfill
	     Por que usamos os valores zero e um para descrever informações qualitativas? Em certo sentido, esses valores são arbitrários: quaisquer dois valores diferentes serviriam. 
		\vfill
		O benefício real de capturar informação qualitativa usando variáveis zero-um é que elas levam a modelos de regressão nos quais os parâmetros têm interpretações bastante naturais, como veremos agora.
        
		\vfill

\end{frame}



\begin{frame}{A Descrição das Informações Qualitativas}
        \justifying
		\vfill
        A tabela a seguir fornece uma listagem parcial de um possível conjunto de dados sobre salários:
		\vfill 
		\footnotesize
		\begin{equation}  
		\begin{array}{|c|c|c|c|c|c|c|}
		\hline \text { pessoa } & {\text { salário} } & {\text { educ }} & {\text { exper }} & {\text { feminino }} & {\text { casado}} \\
		\hline 1 & {3.10} & {11} & {2} & {1} & {0} \\
		\hline 2 & {3.24} & {12} & {22} & {1} & {1} \\
		\hline 3 & {3.00} & {12} & {2} & {0} & {0} \\
		\hline 4 & {6.00} & {8} & {44} & {0} & {1} \\
		\hline 5 & {5.30} & {12} & {7} & {0} & {1} \\
		\hline . & {.} & {.} & {.} & {.} & {.}  \\
		\hline . & {.} & {.} & {.} & {.} & {.}  \\
		\hline 525 & {11.56} & {16} & {5} & {0} & {1}  \\
		\hline 526 & {3.50} & {14} & {5} & {1} & {0} \\
		\hline
		\end{array}\nonumber
		\end{equation}
\end{frame}

\begin{frame}{A Descrição das Informações Qualitativas}
		\justifying
		\vfill 
		Vemos que a Pessoa 1 é do sexo feminino e não é casada, a Pessoa 2 é do sexo feminino e é casada, a Pessoa 3 é do sexo masculino e não é casada, e assim por diante.
		\vfill

\end{frame}


\begin{frame}{Uma Única Variável Dummy Independente}
		\justifying
		\vfill 
		Como incorporamos informações binárias em modelos de regressão? 
		\vfill
		No caso mais simples, com somente uma variável dummy explicativa, simplesmente adicionamos a variável à equação como uma variável independente.
		\vfill
       
        
\end{frame}

\begin{frame}{Uma Única Variável Dummy Independente}
	   
		\begin{tcolorbox}
	Por exemplo, considere o seguinte modelo simples de determinação de salários por hora:
		\begin{equation}
		 { salarioh} =\beta_{0}+ \delta_{0} {feminino} + \beta_{1}  { educ }+u\label{23}
		\end{equation}
		No modelo (1), somente dois fatores observados afetam os salários: gênero e educação. Como feminino = 1 quando a pessoa é mulher e feminino = O quando a pessoa é homem, o parâmetro $\delta_{0}$ tem a seguinte interpretação: $ \delta $ é a diferença no salário por hora entre mulheres e homens, dado o mesmo grau de educação (e o mesmo termo erro $u$). 
			
		\end{tcolorbox}
	\end{frame}



\begin{frame}{Uma Única Variável Dummy Independente} 
        \vfill
		Em termos de expectativas, se assumirmos a hipótese de média condicional zero \mathrm{E}($u$|$feminino$,$educ$) = 0, então 
         \begin{equation}
		\begin{array}{crl}
		 \delta_{0} = 
		\mathrm{E}(salarioh|feminino = 1, educ)
		 \\ \centerline{-} \\
		\mathrm{E}(salarioh|feminino = 0, educ)
		\end{array}\nonumber
		\end{equation} 
		    \justifying
        	Como {\small $feminino$ = 1} corresponde a mulheres e {\small $feminino$ = 0} corresponde a homens, o importante aqui é que o nível de educação é o mesmo em ambas as expectativas; a diferença, $\delta_{0}$, deve-se somente ao gênero.
		
\end{frame}

\begin{frame}{Uma Única Variável Dummy Independente}
		\justifying
		\vfill 
		A situação pode ser descrita graficamente como um deslocamento de intercepto entre as linhas que representam homens e mulheres.
		\vfill
		A diferença não depende do nível de educação, e isso explica a razão de os perfis salário-educação das mulheres e dos homens serem paralelos.
		\vfill
       
\end{frame}


	\begin{frame}{Uma Única Variável Dummy Independente}
	\begin{figure}
		\centering
		\caption{$salarioh =\beta_{0}+ \delta_{0} {feminino} + \beta_{1}educ$ para $\delta_{0}$ < 0.} 
		\includegraphics[width=.825\textwidth]{figuras/fig71.png}
		\label{fig21}
	\end{figure}
	\end{frame}
	
\begin{frame}{Uma Única Variável Dummy Independente}
		\justifying
		\vfill 
		Como podemos efetivamente testar a discriminação salarial? A resposta é simples: simplesmente estimamos o modelo por MQO, exatamente como antes, e usamos a estatística t habitual. 
		\vfill
		Nada muda na mecânica do MQO ou na teoria estatística quando algumas das variáveis independentes são definidas como variáveis dummy. A única diferença em relação ao que vínhamos fazendo até agora é a interpretação do coeficiente da variável dummy.
		\vfill
       
        
\end{frame}





\begin{frame}{Uma Única Variável Dummy Independente}
		
		\begin{tcolorbox}
			\indent\textbf{\small{Exemplo 7.1 (Equação dos Salários por Hora)}}\\
		\begin{equation}
		\begin{split}
		\small{{salarioh} = -1.57 -  1.81 {feminino} + 0.572 { educ }
		\\ \centerline{\scriptsize{$(0.72)$ \hspace{2em} $(0.26)$ \hspace{6em}$(0.049)$}} 
		\\ \centerline{$+ 0.025{exper} + 0.141{perm}$}} 
		\\ \centerline{\scriptsize{$(0.012)$ \hspace{5em} $(0.021)$}}\nonumber
	    \\ \centerline{\footnotesize{$(n = 526)$ \hspace{0.5em} $(R^2 = 0.364)$}}\nonumber
		 \end{split} 
		\end{equation}
		\\ \scriptsize {O intercepto negativo - para os homens, neste caso - não é muito significativo, já que ninguém na amostra tem anos de $educ$, $exper$ e $perm$ próximos de zero. O coeficiente de feminino é interessante pois registra a diferença média no salário por hora entre uma mulher e um homem, dados os mesmos níveis de $educ$, $exper$ e $perm$. Se compararmos uma mulher e um homem com os mesmos níveis de educação, experiência e permanência, a mulher ganha, em média, 1.81 dólares por hora a menos que o homem}.
		\end{tcolorbox}
	\end{frame}

\begin{frame}{Uma Única Variável Dummy Independente}
		\begin{tcolorbox}
			\indent\textbf{\small{Exemplo 7.1 (Continuação)}}\\
		\\ \footnotesize{É esclarecedor comparar o coeficiente de feminino na equação anterior com a estimativa que obtemos quando todas as outras variáveis explicativas são eliminadas da equação}:
		\small
		\begin{equation}
		\begin{split}
		\small{{salarioh} = 7.10 -  2.51 {feminino}}\nonumber
		\end{flushleft}
		\\ \centerline{\footnotesize{$(0.21)$ \hspace{0.5em} $(0.30)$}}\nonumber
		\\ \centerline{\footnotesize{$(n = 526)$ \hspace{0.5em} $(R^2 = 0.116)$}}
        \end{split}
		\footnotesize {Os coeficientes acima têm uma interpretação simples. O intercepto é o salário-hora médio dos homens na amostra ($feminino = 0$), de modo que os homens ganham, em média, $7.1$ dólares por hora. O coeficiente de feminino é a diferença no salário médio entre homens e mulheres. Assim, o salário médio das mulheres, na amostra, é $7.10$ - $2.51$ = $4.59$, ou $4.59$ dólares por hora}.
		\end{tcolorbox}
	\end{frame}


\begin{frame}{Uma Única Variável Dummy Independente}
		\justifying
		\begin{tcolorbox}
			\indent\textbf{\small{A Interpretação dos Coeficientes de Variáveis Dummy Explicativas quando a Variável Dependente é Expressa como $log(y)$}} \\
		\\ \small {Uma especificação comum em trabalhos aplicados tem a variável dependente aparecendo naforma logarítmica, com uma ou mais variáveis dummy aparecendo como variáveis independentes}.\\
		\\ \small{ Como interpretamos os coeficientes das variáveis dummy neste caso?}\\ 
	   	\\ \small{ Não surpreendentemente, os coeficientes têm uma interpretação percentual}.
		\end{tcolorbox}
	\end{frame}

\begin{frame}{Uma Única Variável Dummy Independente}
		\justifying
		\begin{tcolorbox}
			\indent\textbf{\small{A Interpretação dos Coeficientes de Variáveis Dummy Explicativas quando a Variável Dependente é Expressa como $log(y)$}} \\
		\\ \small {De forma geral, se $\hat{\beta_{1}}$ for o coeficiente de uma variável $dummy$, digamos, $x_1$, quando $log(y)$ é a variável dependente, a diferença percentual exata em y previsto quando $x_1$ = $1$ versus quando $x_1 = 0$ é}\\
		\\\centerline{ \small{ $100$ . [exp($\hat{\beta_{1}}) - 1$].}}\\ 
	   	\\ \small{O coeficiente $\hat{\beta_{1}}$ estimado pode ser positivo ou negativo, e é importante preservar seu sinal ao computar}.
		\end{tcolorbox}
	\end{frame}



{
\setbeamerfont{frametitle}{size=\large}
\begin{frame}{Variáveis $Dummy$ para Categorias Múltiplas}
		\justifying
		\vfill 
		Podemos usar diversas variáveis dummy independentes na mesma equação. 
		\vfill
       Por exemplo, estimemos um modelo que considere diferenças salariais entre quatro grupos: homens casados, mulheres casadas, homens solteiros e mulheres solteiras.
		\vfill
       
\end{frame}
} 

{
\setbeamerfont{frametitle}{size=\large}
\begin{frame}{Variáveis $Dummy$ para Categorias Múltiplas}
		\justifying
		\vfill 
		Para fazermos isso temos que selecionar um grupo base; escolhemos homens solteiros. 
		\vfill
       Então, devemos definir as variáveis dummy para cada um dos demais grupos. Vamos chamá-los $hcasados$, $mcasadas$ e $msolteiras$.
       \vfill 
	    Colocando essas três variáveis na equação (e eliminando feminino, já que agora ela é redundante) temos que:
		\vfill
       
\end{frame}
} 

{
\setbeamerfont{frametitle}{size=\large}
\begin{frame}{Variáveis $Dummy$ para Categorias Múltiplas}
		\begin{tcolorbox}
			\indent\textbf{Exemplo 7.6 \small{(Equação do Log do Salário-Hora)}}\\
		\small
		\begin{equation}
		\begin{split}
		\small{{\hat{log(salarioh)}} = 0.321 +  0.213 {hcasados} - 0.198{mcasadas}
		\\ \centerline{\scriptsize{$(0.100)$ \hspace{2em} $(0.055)$ \hspace{6em}$(0.058)$}} 
		\\ \leftline{$ + 0.110{msolteiras} + 0.079{educ} + 0.027{exper}$}
		\\ \centerline{\scriptsize{$(0.056)$ \hspace{6em}$(0.007)$ \hspace{2em} $(0.005)$}}\nonumber
		\\ \leftline{$ - 0.00054{exper^2} + 0.029{perm} + 0.00053{perm^2}$}}
		\\ \centerline{\scriptsize{$(0.00011)$\hspace{3em}$(0.007)$ \hspace{3em} $(0.00023)$}}\nonumber
	    \\ \centerline{\scriptsize{$(n = 526)$ \hspace{1em} $(R^2 = 0.461)$}}\nonumber
		 \end{split} 
		\end{equation}
		\footnotesize {Todos os coeficientes, exceto o de $msolteiras$, têm estatísticas $t$ bem acima de dois, em valores absolutos. A estatística $t$ de $msolteiras$ está em torno de $-1,96$, que é significante apenas ao nível de $5$ contra uma alternativa bilateral}.
	   \end{tcolorbox}

       
\end{frame}
} 

{
\setbeamerfont{frametitle}{size=\large}
\begin{frame}{Variáveis $Dummy$ para Categorias Múltiplas}
		\justifying
		\vfill 
	Para interpretar os coeficientes das variáveis $dummy$, devemos nos lembrar de que o grupo base é o
    de homens solteiros. 
        \vfill
        Assim, as estimativas das três variáveis $dummy$ medem a diferença proporcional nos salários relativamente aos homens solteiros.
       \vfill 
	    
       
\end{frame}
} 

{
\setbeamerfont{frametitle}{size=\large}
\begin{frame}{Variáveis $Dummy$ para Categorias Múltiplas}
		\justifying
		\vfill
       Por exemplo, estima-se que os homens casados ganhem cerca de $21,3\%$ mais que os homens solteiros, mantendo fixas educação, experiência e permanência.
        \vfill
       Uma mulher casada, no entanto, deve ganhar $19,8\%$ menos que um homem solteiro com os mesmos níveis das outras variáveis.
       \vfill 
	    
       
\end{frame}
} 

{
\setbeamerfont{frametitle}{size=\large}
\begin{frame}{Variáveis $Dummy$ para Categorias Múltiplas}
		\justifying
		\vfill
       O exemplo anterior ilustra um princípio geral para a inclusão de variáveis dummy que indicam grupos diferentes:  se o modelo de regressão deve ter diferentes interceptos para, digamos, $g$ grupos ou categorias, precisamos incluir $g - 1$ variáveis dummy no modelo, juntamente com um intercepto.
        \vfill
       O intercepto do grupo base é o intercepto global no modelo, e o coeficiente da variável dummy de um determinado grupo representa a diferença estimada nos interceptos entre aquele grupo e o grupo base.
       \vfill 
	    
       
\end{frame}
} 
 
{
\setbeamerfont{frametitle}{size=\large}
\begin{frame}{Variáveis $Dummy$ para Categorias Múltiplas}
		\justifying
		\vfill
       A inclusão de $g$ variáveis dummy juntamente com um intercepto resultará na armadilha da variável $dummy$.
        \vfill
       Uma alternativa é incluir $g$ variáveis $dummy$ e excluir um intercepto global.
       \vfill 
	   Isso não é recomendável, pois o teste de diferenças relativas a um grupo base se tornará difícil, e alguns programas de regressão alteram a maneira como o $R^2$ é computado quando a regressão não contém um intercepto.
	   \vfill
       
\end{frame}
} 

{
\setbeamerfont{frametitle}{size=\large}
\begin{frame}{Variáveis $Dummy$ para Categorias Múltiplas}
		\begin{tcolorbox}
			\indent\textbf{\small{Incorporação de Informações Ordinais com o Uso de Variáveis Dummy}}\\ 
			\\ \footnotesize{Suponha que gostaríamos de estimar o efeito do risco de crédito das cidades sobre as taxas de juros dos títulos públicos municipais ($TTM$). Suponha ainda que a classificação varie de zero a quatro, na qual zero é o pior risco de crédito e quatro, o melhor. Este é um exemplo de uma \textbf{variável ordinal}, a qual chamaremos de $CR$}.
			\begin{equation}
			TTM = \beta_{0} + \beta_{1}CR + outros fatores,\label{eq1}\nonumber
			\end{equation}
		   \footnotesize{onde deliberadamente não mostramos quais são os outros fatores. Neste caso, $\beta_{1}$ é a mudança em pontos percentuais em $TTM$ quando $CR$ aumenta uma unidade, mantendo fixos todos os outros fatores.}
			
		\end{tcolorbox}
       
\end{frame}
}
 
{
\setbeamerfont{frametitle}{size=\large}
\begin{frame}{Variáveis $Dummy$ para Categorias Múltiplas}
		\begin{tcolorbox}
			\indent\textbf{\small{Incorporação de Informações Ordinais com o Uso de Variáveis Dummy}}\\ 
			\\ \footnotesize{Infelizmente, é bastante difícil interpretar um aumento de uma unidade em $CR$. Sabemos o significado quantitativo de mais um ano de educação, ou de um dólar a mais gasto por aluno, mas fatores como risco de crédito, em geral, têm apenas significado ordinal.}\\
		  \\ \footnotesize{Sabemos que um  $CR$ de quatro é melhor que um $CR$ de três, mas será que a diferença entre quatro e três é a mesma que a diferença entre um e zero? Se não, não fará sentido assumir que um aumento de uma unidade em $CR$ terá um efeito constante sobre $TTM$.}\\
	
		\end{tcolorbox}
       
\end{frame}
} 

{
\setbeamerfont{frametitle}{size=\large}
\begin{frame}{Variáveis $Dummy$ para Categorias Múltiplas}
		\begin{tcolorbox}
			\indent\textbf{\small{Incorporação de Informações Ordinais com o Uso de Variáveis Dummy}}\\ 
			\\ \footnotesize{Uma abordagem melhor que podemos implementar, pois $CR$ assume relativamente poucos valores,é definir variáveis dummy para cada valor de $CR$.}\\	
			\\ \footnotesize{Assim, definimos $CR1 = 1$ se $CR = 1$, e, caso contrário, $CR 1 = 0$, $CR2 = 1$ se $CR = 2$ e, caso contrário, $CR2 = 0$, e assim por diante.}\\
			\\ \footnotesize{ Na realidade, levamos em conta o risco de crédito e o transformamos em cinco categorias. Desta forma podemos estimar o segiunte modelo:}
			
			\begin{equation}
			TTM = \beta_{0} + \delta_{1}CR_1 + \delta_{2}CR_2 + \delta_{3}CR_3 + \delta_{4}CR_4 + outros fatores.\label{eq1}\nonumber
			\end{equation}
		 
			
		\end{tcolorbox}
       
\end{frame}
}

{
\setbeamerfont{frametitle}{size=\large}
\begin{frame}{Variáveis $Dummy$ para Categorias Múltiplas}
		\begin{tcolorbox}
			\indent\textbf{\small{Incorporação de Informações Ordinais com o Uso de Variáveis Dummy}}\\ 
			
		   \footnotesize{Seguimos nossa regra sobre inclusão de variáveis dummy em um modelo, incluindo quatro, já que temos cinco categorias. Omitiu-se a categoria risco de crédito zero, pois ela é o grupo base.} \\
		   \\ \footnotesize{Os coeficientes são de fácil interpretação: $\delta_{1}$ é a diferença em $TTM$ entre uma cidade com risco de crédito um e outra com risco de crédito zero: $\delta_{2}$ é a diferença em $TTM$ entre uma cidade com um risco de crédito dois e outra com risco de crédito zero; e assim por diante.}
			
		\end{tcolorbox}
       
\end{frame}
} 

{
\setbeamerfont{frametitle}{size=\Large}
\begin{frame}{Interações Envolvendo Variáveis $Dummy$}
		\justifying
		\vfill
        Assim como as variáveis com significados quantitativos podem interagir em modelos de regressão, as variáveis $dummy$ também podem. 
        \vfill
        Vimos uma ilustração disso no Exemplo 7.6, no qual definimos quatro categorias com base em estado civil e gênero. 
        \vfill
        Podemos reformular aquele modelo adicionando um termo de interação entre feminino e casado, onde essas variáveis apareçam separadamente. Isso possibilita que o prêmio por ser casado dependa do gênero.
\end{frame}
}  

{
\setbeamerfont{frametitle}{size=\Large}
\begin{frame}{Interações Envolvendo Variáveis $Dummy$}
		\justifying
		\vfill
        Com o propósito de comparação, o modelo estimado com o termo de interação feminino-casado é:\\
       	\begin{equation}
		\begin{split}
		\small{\hat{{log(salario)}} = 0.321 -  0.110 {feminino} + 0.213 {casado}
		\\ \centerline{\scriptsize{$(0.100)$ \hspace{2em} $(0.056)$ \hspace{7em}$(0.055)$}} 
		\\ \centerline{$ + 0.301{feminino*casado} + ...$}} 
		\\ \centerline{\scriptsize{$(0.072)$}}\nonumber
		 \end{split} 
		\end{equation}
		onde o restante da regressão será necessariamente idêntico ao exemplo 7.6. \\
\end{frame}		
}  

{
\setbeamerfont{frametitle}{size=\Large}
\begin{frame}{Interações Envolvendo Variáveis $Dummy$}
		\justifying
		\vfill
       A equação anterior mostra que existe uma interação estatisticamente significante entre gênero e estado civil.  valores zero e um. 
       \vfill
       Este modelo também permite obter o diferencial estimado de salários entre todos os quatro grupos, mas aqui devemos ter o cuidado de inserir a correta combinação de valores zero e um.
        \vfill
      
\end{frame}
}  

{
\setbeamerfont{frametitle}{size=\Large}
\begin{frame}{Interações Envolvendo Variáveis $Dummy$}
		\justifying
       \vfill
       A definição $feminino = 0$ e $casado = 0$ corresponde ao grupo de homens solteiros, que é o grupo base, já que isso elimina $feminino$, $casado$, e $feminino*casado$.
        \vfill
        Podemos encontrar o intercepto de homens casados definindo $feminino = 0$ e $casado = 1$ na regressão; isso produz um intercepto de $0,321 + 0,213 = 0,534$, e assim por diante.
        \vfill
      
\end{frame}
}  

{
\setbeamerfont{frametitle}{size=\Large}
\begin{frame}{Interações envolvendo Variáveis $Dummy$}
		\begin{tcolorbox}
			\indent\textbf{\small{Consideração de Inclinações Diferentes}}\\ 
			\\ \footnotesize{Vimos vários exemplos de como permitir diferentes interceptos para qualquer número de grupos em um modelo de regressão múltipla}.\\
			\\ \footnotesize{Também existem casos de interação de variáveis dummy com variáveis explicativas que não são dummy para permitir uma diferença nas inclinações.}
			\begin{equation}
			log(salarioh) = (\beta_{0} + \delta_{0}feminino) + (\beta_{1} + \delta_{1}feminino)educ + u.\label{eq1}\nonumber
			\end{equation}
		   \footnotesize{Se fizermos $feminino = 0$, veremos que o intercepto de homens é $\beta_{0}$, enquanto a inclinação na educação dos homens é $\beta_{1}$. Para as mulheres, usamos $feminino = 1$; assim, o intercepto para as mulheres será $\beta_{0} + \delta_{0}$ e a inclinação será $\beta_{1} + \delta_{1}$ Portanto, $\delta_{0}$ mede a diferença nos interceptos entre mulheres e homens, enquanto ô 1 mede a diferença no retomo da educação entre mulheres e homens.}
			
		\end{tcolorbox}
       
\end{frame}
} 
 
{
\setbeamerfont{frametitle}{size=\Large}
\begin{frame}{Interações Envolvendo Variáveis $Dummy$}
	 \begin{tcolorbox}
	   \begin{figure}
		\centering
		\caption{\footnotesize{Gráficos da equação anterior. (a) $\delta_{0} < 0$, $\delta_{1} < 0$ ; (b) $\delta_{0}< 0$, $\delta_{1} > 0.$}}
		\includegraphics[width=.825\textwidth]{figuras/figura cap 7 wooldrdige estagio doc.png}
		\label{fig21}
	\end{figure}
			
		\end{tcolorbox}
       
\end{frame}
}  

{
\setbeamerfont{frametitle}{size=\Large}
\begin{frame}{Interações envolvendo Variáveis $Dummy$}
		\begin{tcolorbox}
			\indent\textbf{\small{Consideração de Inclinações Diferentes}}\\ 
			\\ \footnotesize{Dois dos quatro casos dos sinais de $\delta_{0}$ e $\delta_{1}$ são apresentados na figura anterior. O gráfico (a) mostra o caso em que o intercepto das mulheres está abaixo do intercepto dos homens, enquanto a inclinação da linha é menor para as mulheres do que para os homens. Isso significa que as mulheres ganham menos que os homens em todos os níveis de educação, e a diferença aumenta conforme educ se toma maior}.\\
			\\ \footnotesize{No gráfico (b), o intercepto das mulheres está abaixo do intcrcepto
            dos homens, mas a inclinação da educação é maior para as mulheres. Isso significa que as mulheres ganham menos que os homens em baixos níveis de educação, mas a diferença diminui conforme a educação aumenta. Em algum ponto, uma mulher ganhará mais que um homem, dado o mesmo nível de educação (e esse ponto é facilmente encontrado, dada a equação estimada).}
			
			
		\end{tcolorbox}
       
\end{frame}
}  

{
\setbeamerfont{frametitle}{size=\Large}
\begin{frame}{Interações envolvendo Variáveis $Dummy$}
		\begin{tcolorbox}
			\indent\textbf{\small{Consideração de Inclinações Diferentes}}\\ 
			\\ \footnotesize{Como podemos estimar o modelo da última equação? Para aplicar o MQO, devemos escrever o modelo com uma interação entre $feminino$ e $educ$}:\\
			\begin{equation}
			log(salarioh) = \beta_{0} + \delta_{0}feminino + \beta_{1}educ + \delta_{1}feminino*educ + u.\label{eq1}\nonumber
			\end{equation}
			\\ \footnotesize{Os parâmetros agora podem ser estimados a partir da regressão de log(salárioh) sobre $feminino,educ$, e $feminino*educ$. A obtenção do termo de interação é fácil com o uso de qualquer programa de regressão.}\\
		    \\ \footnotesize{Nesta última equação, precisamos usar um teste F para testar $H_0 : \delta_{0} = 0, \delta_{1} = 1$. No modelo com apenas uma diferença de interceptos, rejeitamos essa hipótese, pois $H_0: \delta_{0} = 0$ é completamente rejeitada contra $H_0: \delta_{0} < 0$}.
		\end{tcolorbox}
       
\end{frame}
}


{
\setbeamerfont{frametitle}{size=\Large}
\begin{frame}{Interações envolvendo Variáveis $Dummy$}
		\begin{tcolorbox}
			\indent\textbf{\small{Verificação de Diferenças nas Funções de Regressão entre Grupos}}\\ 
			\\ \footnotesize{Os exemplos anteriores ilustram que a interação de variáveis $dummy$ com outras variáveis independentes pode ser uma ferramenta poderosa.}\\
			\\ \footnotesize{Algumas vezes, queremos testar a hipótese nula de que duas populações, ou grupos, seguem a mesma função de regressão, contra a hipótese alternativa de que uma ou mais das inclinações diferem entre os grupos.}\\
		    \\ \footnotesize{Suponha que queiramos testar se o mesmo modelo de regressão descreve a nota média no curso superior de atletas universitários masculinos e femininos. A equação é} 
		    	\begin{equation}
			nmgradac = \beta_{0} + \beta_{1}sat + \beta_{2}emperc + \beta_{3}tothrs + u,\label{eq1}\nonumber
			\end{equation}
		\end{tcolorbox}
       
\end{frame}
}  

{
\setbeamerfont{frametitle}{size=\large}
\begin{frame}{Interações envolvendo Variáveis $Dummy$}
		\begin{tcolorbox}
			\indent\textbf{\footnotesize{Verificação de Diferenças nas Funções de Regressão entre Grupos (continuação)}}\\ 
			\\ \footnotesize{onde $sat$ é a nota obtida no exame de ingresso em curso superior, $emperc$ é o percentil da classificação no ensino médio, e $tothrs$ é o total de horas do curso superior.}\\
			\\ \footnotesize{Sabemos que para considerar uma diferença nos interceptos podemos incluir uma variável $dummy$ para $masculino$ ou $feminino$.}\\
		    \\ \footnotesize{Se estivermos interessados em verificar se existe qualquer diferença entre homens e mulheres, então devemos admitir um modelo no qual o intercepto e todas as inclinações possam ser diferentes entre os grupos:}
		    \begin{equation}
		    \begin{split}
			\RaggedRight{nmgradac = \beta_{0} + \delta_{0}feminino + \beta_{1}sat + \delta_{1}feminino*sat \\ 
			+ \beta_{2}emperc 
			+ \delta_{2}feminino*emperc + \beta_{2}emperc + \\
			\beta_{3}tothrs + \delta_{3}feminino*tothrs + u}.\label{eq1}\nonumber
			\end{split}
			\end{equation}
		\end{tcolorbox}
       
\end{frame}
} 

{
\setbeamerfont{frametitle}{size=\Large}
\begin{frame}{Interações envolvendo Variáveis $Dummy$}
		\begin{tcolorbox}
			\indent\textbf{\small{Verificação de Diferenças nas Funções de Regressão entre Grupos}}\\ 
			\\ \footnotesize{O parâmetro $\delta_{0}$ é a diferença nos interceptos entre mulheres e homens, $\delta_{1}$ é a diferença de inclinações em relação a $sat$ entre mulheres e homens, e assim por diante.}\\
			\\ \footnotesize{A hipótese nula de que $nmgradac$ segue o mesmo modelo para homens e mulheres é escrita como}\\
		    	\begin{equation}
			 H_0 : \delta_{0} = 0 , \delta_{1} = 0 , \delta_{2} = 0 , \delta_{3} = 0. \nonumber 
			 \end{equation}
			 \\ \footnotesize{Se um dos $\delta_{j}$ for diferente de zero, então os modelos são diferentes para homens e mulheres.}
		\end{tcolorbox}
       
\end{frame}
}  

{
\setbeamerfont{frametitle}{size=\Large}
\begin{frame}{Interações envolvendo Variáveis $Dummy$}
		\begin{tcolorbox}
			\indent\textbf{\small{Verificação de Diferenças nas Funções de Regressão entre Grupos}}\\ 
			\\ \footnotesize{No modelo geral com k variáveis explicativas e um intercepto, suponha que temos dois grupos, que chamaremos de $g = 1$ e $g = 2$.}\\
		    \begin{equation}
			y = \beta_{g,0} + \beta_{g,1}x_1 + \beta_{g,2}x_2 + ... + \beta_{g,k}x_k + u \label{eq1}\nonumber
			\end{equation}
			 \\ \footnotesize{para $g = 1$ e $g = 2$. A hipótese de que cada beta é o mesmo nos dois grupos envolve $k + 1$ restrições (no exemplo de $nmgradac$, $k + 1 = 4$)}.\\
			 \\ \footnotesize{O modelo sem restrições, que pode ser entendido como tendo uma variável $dummy$ de grupo e $k$ termos de interação, além do intercepto e das próprias variáveis, tem $n - 2(k + 1)$ graus de liberdade}.
		\end{tcolorbox}
       
\end{frame}
}  

{
\setbeamerfont{frametitle}{size=\Large}
\begin{frame}{Interações envolvendo Variáveis $Dummy$}
		\begin{tcolorbox}
			\indent\textbf{\small{Verificação de Diferenças nas Funções de Regressão entre Grupos}}\\ 
			\\ \footnotesize{A percepção básica é que a soma dos resíduos quadrados do modelo sem restrições pode ser obtida de duas regressões separadas, uma para cada grupo.}\\
			 \\ \footnotesize{Seja $SQR_1$ a soma dos resíduos quadrados obtida ao estimar para o primeiro grupo; isso envolve $n_1$ observações.Seja $SQR_2$ a soma dos resíduos quadrados obtida ao estimar o modelo usando o segundo grupo ($n_2$ observações)}.\\
			 \\ \footnotesize{A soma dos resíduos quadrados do modelo sem restrições é simplesmente $SQR_i_r = SQR_1 + SQR_2$, i.e., a soma dos resíduos quadrados restrita é somente a $SQR$ do agrupamento dos grupos e da estimativa de uma única equação, digamos $SQR_p$}.
			 \begin{equation}
			y = \beta_{g,0} + \beta_{g,1}x_1 + \beta_{g,2}x_2 + ... + \beta_{g,k}x_k + u \label{eq1}\nonumber
			\end{equation}
		\end{tcolorbox}
       
\end{frame}
}  
{
\setbeamerfont{frametitle}{size=\Large}
\begin{frame}{Interações envolvendo Variáveis $Dummy$}
		\begin{tcolorbox}
			\indent\textbf{\small{Verificação de Diferenças nas Funções de Regressão entre Grupos}}\\ 
			\\ \footnotesize{Uma vez calculados esses termos, computamos a estatística $F$ da forma habitual:}\\
			 \small
			 \\ \centerline{$F$ = {[$SQR_p$ - ($SQR_1 + SQR_2$)]/ ($SQR_1 + SQR_2$)}}\\
			\\ \centerline{*} \\
			 \\ \centerline{ [$n - 2(k + 1)$]/($k + 1$)}\\ 
		\\ \footnotesize{onde $n$ é o número total de observações. Esta estatística $F$ específica é usualmente chamada em econometria de  \textbf{estatística de Chow.} Como o teste de Chow é apenas um teste F, ele só é válido sob homoscedasticidade. Em particular, sob a hipótese nula, as variâncias dos erros dos dois grupos devem ser iguais}.
	
		\end{tcolorbox}
       
\end{frame}
}  

{
\setbeamerfont{frametitle}{size=\Large}
\begin{frame}{Interações envolvendo Variáveis $Dummy$}
		\begin{tcolorbox}
			\indent\textbf{\small{Verificação de Diferenças nas Funções de Regressão entre Grupos}}\\ 
			\\ \footnotesize{Como sempre, a normalidade não é necessária para a análise assimptótica.}\\
		\\ \footnotesize{Uma limitação importante do teste de Chow, independentemente do método usado para implementá-lo, é a hipótese nula não permitir nenhuma diferença entre os grupos. Em muitos casos, é mais interessante considerar uma diferença nos interceptos entre os grupos e depois verificar as diferenças das inclinações}.\\
	    \\ \footnotesize{Há duas maneiras de fazermos com que os interceptos difiram sob a hipótese nula}.\\
	  
		\end{tcolorbox}
       
\end{frame}
}  
{
\setbeamerfont{frametitle}{size=\Large}
\begin{frame}{Interações envolvendo Variáveis $Dummy$}
		\begin{tcolorbox}
			\indent\textbf{\small{Verificação de Diferenças nas Funções de Regressão entre Grupos}}\\ 
		 \\\footnotesize{Uma delas é incluir a dummy do grupo e todos os termos de interação, mas apenas testar a significância conjunta dos termos de interação.}\\	
		\\ \footnotesize{ A segunda é calcular uma estatística $F$ (como na última equação), mas onde a soma de quadrados restrita, chamada $SQR_p$, é obtida pela regressão que permite somente um deslocamento do intercepto}.\\
	    \\ \footnotesize{Em outras palavras, computamos uma regressão agrupada e apenas incluímos as variáveis $dummy$ que distinguem os dois grupos.}\\
		\end{tcolorbox}
       
\end{frame}
}  

{
\setbeamerfont{frametitle}{size=\large}
\begin{frame}{Uma Variável Dependente Binária: O Modelo de Probabilidade Linear}
		\justifying
       \vfill
      Nas últimas seções, estudamos como podemos incorporar informações qualitativas, por exemplo, variáveis explicativas em um modelo de regressão múltipla, por meio do uso de variáveis independentes binárias.
        \vfill
       Em todos os modelos vistos até agora, a variável dependente $y$ teve um significado $quantitativo$ ($y$ é um montante em dólares, uma pontuação em um teste, uma percentagem, ou seus $logs$, etc).
      
        \end{frame}
}  

{
\setbeamerfont{frametitle}{size=\large}
\begin{frame}{Uma Variável Dependente Binária: O Modelo de Probabilidade Linear}
		\justifying
       \vfill
        O que acontece se quisermos usar regressão múltipla para $explicar$ um evento qualitativo?
        \vfill
        No caso mais simples o evento que gostaríamos de explicar - e que aparece com muita frcqüência na prática - é um resultado binário.
        \vfill
        Em outras palavras, nossa variável dependente, $y$, assume somente um dos dois valores: zero ou um.
        \end{frame}
}  

{
\setbeamerfont{frametitle}{size=\large}
\begin{frame}{Uma Variável Dependente Binária: O Modelo de Probabilidade Linear}
		\justifying
       \vfill
       Por exemplo, $y$ pode ser definido para indicar se um adulto concluiu o ensino médio;
        \vfill
        $y$ pode indicar se um aluno do curso superior usou drogas ilegais durante determinado ano escolar;
        \vfill
        ou y pode indicar se uma empresa foi absorvida por outra durante determinado ano.
        \vfill
        Em cada um desses exemplos, podemos definir que $y = 1$ represente um dos resultados e $y = 0$, o outro.
        \end{frame}
}  

{
\setbeamerfont{frametitle}{size=\large}
\begin{frame}{Uma Variável Dependente Binária: O Modelo de Probabilidade Linear}
		\justifying 
        \vfill
        Isso significaria escrever um modelo de regressão múltipla, tal como:
        \begin{equation}
        y = \beta_{0} + \beta_{1}x_1 + ... + \beta_{k}x_k + u, \nonumber
        \end{equation}
        quando $y$ for uma variável binária? Como $y$ pode assumir somente dois valores, $\beta_{j}$ não pode ser interpretado como a mudança em $y$ devido ao aumento de uma unidade em $x_j$, mantendo fixos todos os
        outros fatores: $y$ somente muda de zero para um ou de um para zero. No entanto, os coeficientes $\beta_{j}$ ainda têm interpretações úteis.
        \end{frame}
}  

{
\setbeamerfont{frametitle}{size=\large}
\begin{frame}{Uma Variável Dependente Binária: O Modelo de Probabilidade Linear}
		\justifying 
         Se assumirmos que a hipótese de média condicional zero RLM.3 é válida, i.e., $E(u|x_1, ... ,x_k) = 0$, então teremos, como sempre,
        \begin{equation}
        E(y|x) = \beta_{0} + \beta_{1}x_1 + ... + \beta_{k}x_k, \nonumber
        \end{equation}
        onde $x$ é uma forma abreviada que representa todas as variáveis explicativas.
        \vfill
        O ponto principal é que, quando $y$ é uma variável binária assumindo os valores zero e um, é sempre verdade que $P(y = 1|x) = E(y|x)$: a probabilidade de "sucesso" - i.e., a probabilidade de que $y = 1$ - é a mesma do valor esperado de $y$. Assim, temos a importante equação
        
        \end{frame}
}  
   
{
\setbeamerfont{frametitle}{size=\large}
\begin{frame}{Uma Variável Dependente Binária: O Modelo de Probabilidade Linear}
		\justifying 
        \begin{equation}
        P(y = 1|x) = \beta_{0} + \beta_{1}x_1 + ... + \beta_{k}x_k, \nonumber
        \end{equation}
       que mostra a probabilidade de sucesso, digamos, $p(x) = P(y = 1 |x)$, uma função linear de $x_j$. 
        \vfill 
        A equação acima é um exemplo de modelo de resposta binária, e $P(y = 1 |x)$  \textbf{também é chamado de probabilidade de resposta}. Como a soma das probabilidades deve ser um, $P(y = O|x) = 1 - P(y = 1 |x)$ também é uma função linear de $x_j$.
        \end{frame}
}  	

{
\setbeamerfont{frametitle}{size=\large}
\begin{frame}{Uma Variável Dependente Binária: O Modelo de Probabilidade Linear}
		\justifying 
		\vfill
         O modelo de regressão linear múltipla com uma variável dependente binária é chamado de \textbf{modelo de probabilidade linear (MPL)} porque a probabilidade de resposta é linear nos parâmetros $\beta_j$· No MPL, $\beta_j$ mede a mudança na probabilidade de sucesso quando $x_j$ muda, mantendo fixos os outros fatores: 
        \begin{equation}
        \Delta P(y = 1|x) = \beta_j \Delta x_j . \nonumber
        \end{equation}
         Com isso em mente, o modelo de regressão múltipla pode nos permitir estimar o efeito de diversas variáveis explicativas sobre eventos qualitativos. A mecânica do MQO é a mesma de antes. 
        \end{frame}
}  	


{
\setbeamerfont{frametitle}{size=\large}
\begin{frame}{Uma Variável Dependente Binária: O Modelo de Probabilidade Linear}
		\justifying 
		\vfill
        Se escrevermos a equação estimada como 
        \begin{equation}
        \hat{y} = \hat{\beta_0} + \hat{\beta_1} x_1 + ... + \hat{\beta_k} x_k , \nonumber
        \end{equation} 
        \vfill 
        temos que nos lembrar que $y$ é a probabilidade de sucesso prevista. Portanto, $\hat{\beta_0}$ é a probabilidade de sucesso prevista quando cada $x_j$ é definido como zero, o que pode, ou não, ser interessante. O coeficiente de inclinação $\hat{\beta_1}$ mede a mudança prevista na probabilidade de sucesso quando $x_1$ aumenta em uma unidade. 

        \end{frame}
}  	



{
\setbeamerfont{frametitle}{size=\large}
\begin{frame}{Conclusões}
		\justifying 
		\vfill
       Neste capítulo aprendemos como usar informações qualitativas na análise de regressão. 
       \vfill
       No caso mais simples, uma variável $dummy$ é definida para fazer a distinção entre dois grupos, e o coeficiente estimado da variável dummy estima a diferença 0ceteris paribus entre os dois grupos. 
        \vfill 
       As variáveis $dummy$ também são úteis para incorporar infornações ordinais, como classificações de crédito e de aparência pessoal, em modelos de regressão. 
       

        \end{frame}
}  	

{
\setbeamerfont{frametitle}{size=\large}
\begin{frame}{Conclusões}
		\justifying 
		\vfill
        Simplesmente definimos um conjunto de variáveis dummy representando os diferentes resultados da variável ordinal, admitindo uma das categorias como grupo base. 
       \vfill
       Para possibilitar diferenças de inclinações entre os diferentes grupos, as variáveis $dummy$ podem interagir com variáveis quantitativas. 
        \vfill 
       No caso extremo, podemos permitir que cada grupo tenha sua própria inclinação em todas as variáveis, como também seu próprio intercepto. 
       

        \end{frame}
}  	

{
\setbeamerfont{frametitle}{size=\large}
\begin{frame}{Conclusões}
		\justifying 
		\vfill
        O teste de Chow pode ser usado para detectar se existem quaisquer diferenças entre os grupos. Em muitos casos, é mais interessante verificar se, após termos permitido uma diferença de interceptos, as inclinações de dois grupos diferentes são as mesmas.
       \vfill
       Um teste F padrão pode ser usado para esse propósito em um modelo irrestrito que inclua interações entre a dummy do grupo e todas as variáveis.
        \vfill 
      
       

        \end{frame}
}  	

{
\setbeamerfont{frametitle}{size=\large}
\begin{frame}{Conclusões}
		\justifying 
		\vfill
        O modelo de probabilidade linear, que é simplesmente estimado pelo MQO, possibilita explicar uma resposta binária usando análise de regressão. 
       \vfill
      As estimativas MQO agora são interpretadas como alterações na probabilidade de "sucesso" $(y = 1)$, dado um aumento de uma unidade na variável explicativa correspondente.
        \vfill 
      
       

        \end{frame}
}  	

{
\setbeamerfont{frametitle}{size=\large}
\begin{frame}{Conclusões}
		\justifying 
		\vfill
        O \textbf{MPL} tem algumas inconveniências: pode produzir probabilidades previstas menores que zero ou maiores que um, implica um efeito marginal constante de cada variável explicativa que aparece em sua forma original, e contém heteroscedasticidade.
       \vfill 
       Os primeiros dois problemas muitas vezes não são graves quando estamos obtendo estimativas dos efeitos parciais das variáveis explicativas na faixa intermediária dos dados.
        \vfill 
        A heteroscedasticidade invalida os erros-padrão usuais do MQO e as estatísticas de testes mas, como veremos no próximo capítulo, isso é facilmente corrigido em amostras suficientemente grandes.
       

        \end{frame}
}  	

	
\end{document}